%----------------------------------------------------------------------------------------
%	PACKAGES AND DOCUMENT CONFIGURATIONS
%----------------------------------------------------------------------------------------

\documentclass{article}

\usepackage[version=3]{mhchem} % Package for chemical equation typesetting
\usepackage{siunitx} % Provides the \SI{}{} and \si{} command for typesetting SI units
\usepackage{graphicx} % Required for the inclusion of images
\usepackage{natbib} % Required to change bibliography style to APA
\usepackage{amsmath} % Required for some math elements 
\usepackage{amssymb}

\setlength\parindent{0pt} % Removes all indentation from paragraphs

\renewcommand{\labelenumi}{\alph{enumi}.} % Make numbering in the enumerate environment by letter rather than number (e.g. section 6)

%\usepackage{times} % Uncomment to use the Times New Roman font

\newcommand{\PartDiv}[1]{\frac{\partial}{\partial #1}}
\allowdisplaybreaks


%----------------------------------------------------------------------------------------
%	DOCUMENT INFORMATION
%----------------------------------------------------------------------------------------

\title{Machine Learning 1 \\ Exercise 6} % Title

\author{Group: BSSBCH} % Author name

\date{\today} % Date for the report


\begin{document}

\maketitle % Insert the title, author and date
\noindent\rule[0.5ex]{\linewidth}{1pt}
Matthias Bigalke, 339547, maku@win.tu-berlin.de \\
Tolga Buz, 346836, buz\_tolga@yahoo.de \\
Alejandro Hernandez, 395678, alejandrohernandezmunuera@gmail.com \\
Aitor Palacios Cuesta, 396276, aitor.palacioscuesta@campus.tu-berlin.de \\
Christof Schubert, 344450, christof.schubert@campus.tu-berlin.de \\
Daniel Steinhaus, 342563, dany.steinhaus@googlemail.com\\
\noindent\rule[0.5ex]{\linewidth}{1pt}
% If you wish to include an abstract, uncomment the lines below
% \begin{abstract}
% Abstract text
% \end{abstract}

%----------------------------------------------------------------------------------------
%	SECTION 1
%----------------------------------------------------------------------------------------

\section*{Exercise 1}

\subsection*{Finding the direction of maximal correlation between datasets}

\section*{Exercise 2}
\subsection*{Fisher and Bayes}
\subsubsection*{(a)}

\subsubsection*{(b)}

We know that two classes are generated by two $d$-dimensional Gaussian distributions $p(x, \omega_1 ) \sim N (\mu_1 , \Sigma_1 )$ and $p(x, \omega_2 ) \sim N (\mu_2 , \Sigma_2 )$ with non-equal covariance matrices ($\Sigma_1 \neq \Sigma_2$).
The covariance matrices being non-equal means that the ratio between the likelihoods between both classes (which we use for classification) is not linear. Instead, we need to use a quadratic discriminant approach (QDA), as the decision boundaries are quadratic. \\ 

Then we can derive from the Bayes approach: \\ 
$b(x) = 1$ if $r_2^2 < r_1^2 + 2\ log \frac{P (\omega_1)}{P(\omega_2)} + log \frac{|\Sigma_1|}{|\Sigma_2|}$ \\
and $b(x) = 0$ otherwise. \\

In this case, $r_i^2 = (x − \mu_i )^T \Sigma_i^{−1} (x − \mu_i )$ with $i = 1, 2$ is the Mahalanobis distance. 
In analogy with the Fisher function, we can thus form the function: \\
$\psi (x) = r_2^2 - r_1^2 - log \frac{|\Sigma_1|}{|\Sigma_2|}$

\end{document}
