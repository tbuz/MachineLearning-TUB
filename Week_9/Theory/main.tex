%----------------------------------------------------------------------------------------
%	PACKAGES AND DOCUMENT CONFIGURATIONS
%----------------------------------------------------------------------------------------

\documentclass{article}

\usepackage[version=3]{mhchem} % Package for chemical equation typesetting
\usepackage{siunitx} % Provides the \SI{}{} and \si{} command for typesetting SI units
\usepackage{graphicx} % Required for the inclusion of images
\usepackage{natbib} % Required to change bibliography style to APA
\usepackage{amsmath} % Required for some math elements 
\usepackage{amssymb}
\usepackage{amsfonts}

\setlength\parindent{0pt} % Removes all indentation from paragraphs

\renewcommand{\labelenumi}{\alph{enumi}.} % Make numbering in the enumerate environment by letter rather than number (e.g. section 6)

%\usepackage{times} % Uncomment to use the Times New Roman font

\newcommand{\PartDiv}[1]{\frac{\partial}{\partial #1}}
\newcommand{\F}[0]{\mathbb{F}}
\newcommand{\N}[0]{\mathbb{N}}
\newcommand{\Z}[0]{\mathbb{Z}}
\newcommand{\Q}[0]{\mathbb{Q}}
\newcommand{\R}[0]{\mathbb{R}}
\newcommand{\C}[0]{\mathbb{C}}
\newcommand{\B}[0]{\mathbb{B}}
\newcommand{\im}[0]{\mathit{i}}
\newcommand{\uber}[2]{{{#1} \choose {#2}}}
\newcommand{\vect}[1]{\begin{pmatrix}#1\end{pmatrix}}
\newcommand{\norm}[1]{\left\lvert #1 \right\rvert}
\allowdisplaybreaks


%----------------------------------------------------------------------------------------
%	DOCUMENT INFORMATION
%----------------------------------------------------------------------------------------

\title{Machine Learning 1 \\ Exercise 9} % Title

\author{Group: BSSBCH} % Author name

\date{\today} % Date for the report


\begin{document}

\maketitle % Insert the title, author and date
\noindent\rule[0.5ex]{\linewidth}{1pt}
Matthias Bigalke, 339547, maku@win.tu-berlin.de \\
Tolga Buz, 346836, buz\_tolga@yahoo.de \\
Alejandro Hernandez, 395678, alejandrohernandezmunuera@gmail.com \\
Aitor Palacios Cuesta, 396276, aitor.palacioscuesta@campus.tu-berlin.de \\
Christof Schubert, 344450, christof.schubert@campus.tu-berlin.de \\
Daniel Steinhaus, 342563, dany.steinhaus@googlemail.com\\
\noindent\rule[0.5ex]{\linewidth}{1pt}
% If you wish to include an abstract, uncomment the lines below
% \begin{abstract}
% Abstract text
% \end{abstract}

%----------------------------------------------------------------------------------------
%	SECTION 1
%----------------------------------------------------------------------------------------

\section*{Exercise 1}

\subsection*{The Dual SVM}

\subsection*{a)}

Lagrange function $\Lambda(w, \theta, \alpha)$ for the hard margin SVM: \\
\begin{align}
    \Lambda(w,\theta ,\alpha) = \frac{1}{2} ||w||^2 - \sum_{i=1}^N \alpha_i(y_i(w^T \Phi(x_i) + \theta)-1)
\end{align}

\subsection*{b)}
The Lagrange dual:
\begin{align}
    \nabla_w\Lambda(w,\theta,\alpha)=w-\sum_{i=1}^N \alpha_i y_i \Phi(x_i) \stackrel{!}{=} 0\\
    \Longleftrightarrow w = \sum_{i=1}^N \alpha_i y_i \Phi(x_i) ~~~~~~ \\
    \nabla_\theta \Lambda(w,\theta,\alpha) = - \sum_{i=1}^N \alpha_i y_i \stackrel{!}{=} 0
\end{align}

Substituting $w = \sum_{i=1}^N \alpha_i y_i \Phi(x_i)$ into equation (1):
\begin{align}
    \Lambda(w,\theta,\alpha) = \frac{1}{2} \bigg|\bigg|{\sum_{i=1}^N \alpha_i y_i \Phi(x_i)}\bigg|\bigg|^2 - \sum_{i=1}^N \alpha_i y_i[\sum_{i=1}^N \alpha_i y_i]^T \cdot \Phi(x_i) - \sum_{i=1}^N \alpha_i y_i \theta + \sum_{i=1}^N \alpha_i \\
    = -\frac{1}{2} \sum_{i=1}^N \sum_{j=1}^N \alpha_i \alpha_j y_i y_j k(x_i, x_j) - \theta \sum_{i=1}^N \alpha_i y_i + \sum_{i=1}^N \alpha_i ~~~~~~~~~~~~~~~
\end{align}

Using $\sum_{j=1}^N \alpha_i y_i = 0$: 
\begin{align}
    \tilde{\Lambda}(\alpha) = - \frac{1}{2}  \sum_{i=1}^N \sum_{j=1}^N \alpha_i \alpha_j y_i y_j k(x_i, x_j) + \sum_{i=1}^N \alpha_i
\end{align}

Following the constraint from (4):
\begin{align}
    \sum_{i=1}^N \alpha_i y_i \stackrel{!}{=} 0
\end{align}

And all $\alpha$ must be positive, thus:
\begin{align}
    \alpha_i \geq 0 \ \forall i \in 1, \dots, N
\end{align}

This shows the dual problem, for which the Lagrange function $\tilde{\Lambda}$ is maximized w.r.t. $\alpha$. \\

In the next step, we can show how a solution of the dual problem can be used to solve the primal problem. $\alpha^\ast$ denotes the optimal solutions of $\alpha$. Using (3):
\begin{align}
    w^\ast = \sum_{i=1}^N \alpha_i^\ast y_i \Phi(x_i)
\end{align}

To obtain $b^\ast$, we use the fact that for any support vector $s$ the hyperplane equation must be equal to 1:
\begin{align}
    y_s \bigg(\sum_{i=1}^N \alpha^\ast y_i k(x_i, s) \bigg) + \theta^\ast = 1 \\
    \Longleftrightarrow \theta^\ast = y_s - \bigg( \sum_{i=1}^N \alpha^\ast y_i k(x_i, s) \bigg)
\end{align}

\subsection*{c)}

\section*{Exercise 2}

\subsection*{SVMs and Quadratic Programming}

\subsection*{a)}


\section*{Exercise 3}

\subsection*{Programming}

Please see the other upload for our solution.

\end{document}
