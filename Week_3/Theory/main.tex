%----------------------------------------------------------------------------------------
%	PACKAGES AND DOCUMENT CONFIGURATIONS
%----------------------------------------------------------------------------------------

\documentclass{article}

\usepackage[version=3]{mhchem} % Package for chemical equation typesetting
\usepackage{siunitx} % Provides the \SI{}{} and \si{} command for typesetting SI units
\usepackage{graphicx} % Required for the inclusion of images
\usepackage{natbib} % Required to change bibliography style to APA
\usepackage{amsmath} % Required for some math elements 
\usepackage{amssymb}

\setlength\parindent{0pt} % Removes all indentation from paragraphs

\renewcommand{\labelenumi}{\alph{enumi}.} % Make numbering in the enumerate environment by letter rather than number (e.g. section 6)

%\usepackage{times} % Uncomment to use the Times New Roman font

%----------------------------------------------------------------------------------------
%	DOCUMENT INFORMATION
%----------------------------------------------------------------------------------------

\title{Machine Learning 1 \\ Exercise 3} % Title

\author{Group: BSSBCH} % Author name

\date{\today} % Date for the report

\begin{document}

\maketitle % Insert the title, author and date
\noindent\rule[0.5ex]{\linewidth}{1pt}
Matthias Bigalke, 339547, maku@win.tu-berlin.de \\
Tolga Buz, 346836, buz\_tolga@yahoo.de \\
Alejandro Hernandez, 395678, alejandrohernandezmunuera@gmail.com \\
Aitor Palacios Cuesta, 396276, aitor.palacioscuesta@campus.tu-berlin.de \\
Christof Schubert, 344450, christof.schubert@campus.tu-berlin.de \\
Daniel Steinhaus, 342563, dany.steinhaus@googlemail.com\\
\noindent\rule[0.5ex]{\linewidth}{1pt}
% If you wish to include an abstract, uncomment the lines below
% \begin{abstract}
% Abstract text
% \end{abstract}

%----------------------------------------------------------------------------------------
%	SECTION 1
%----------------------------------------------------------------------------------------

\section{Maximum Likelihood vs. Bayes}

\textit{An unfair coin is tossed seven times and the event (head or tail) is recorded at each iteration.  The observed sequence of events is:}

\begin{equation}
\mathcal{D} = (x_1, x_2, ... , x_7) = (\text{head, head, tail, tail, head, head, head})
\end{equation}

\textit{We assume that all tosses $x_1, x_2, ...$ have been generated independently following the Bernoulli probability distribution}

\begin{equation}
P(x|\theta) = \left\{
\begin{array}{ll} \theta 	& \text{if} ~ x = \text{head}\\
  			   1 - \theta &  \text{if} ~ x = \text{tail}
\end{array}\right.
\end{equation}

where $\theta \in [0,1]$ is an unknown parameter.

\subsection*{Task a)}

\textit{State the likelihood function $P(\mathcal{D}|\theta)$ that depends on the parameter $\theta$}.

The likelihood function can be calculated with

\begin{align}
\begin{aligned}
P(\mathcal{D}|\theta) = \prod_{i = 1}^7(P(x_i|\theta) = \theta^5 \cdot (1 - \theta)^2 
\end{aligned}
\end{align}

\subsection*{Task b)}

\textit{Compute the maximum likelihood solution $\hat\theta$, and evaluate for this parameter the probability that the next two tosses are "head", that is, evaluate:}

\begin{equation}
P(x_8 = \text{head}, x_9 = \text{head} | \hat\theta)
\end{equation}

The maximum likelihood estimate is $\hat\theta^{ML} = arg \underset{\theta}{max}P(\mathcal{D}|\theta)$ . To find the maximum the maximum likelihood function is derivated:

\begin{align}
\begin{aligned}
\frac{\partial P(\mathcal{D}|\theta)}{\partial \theta} &= \frac{\partial}{\partial \theta} \theta^5 \cdot (1 - \theta)^2\\
									    &= 5 \cdot \theta^4 \cdot (1-\theta)^2 + \theta^5 \cdot (-1) \cdot 2 \cdot (1-\theta)\\
									    &= 5 \cdot \theta^4 \cdot (1 - 2\theta + \theta^2) - 2\theta^5 \cdot (1-\theta)\\
									    &= 5 \theta^4 - 10\theta^5 + 5\theta^6 - 2\theta^5 + 2\theta^6\\
									    &= 7\theta^6 - 12 \theta^5 + 5 \theta^4
\end{aligned}
\end{align}

The derivation has to be zero to find a maximum:

\begin{align}\label{eq:condmaximum}
\begin{aligned}
				&\frac{\partial P(\mathcal{D}|\theta)}{\partial \theta} &\overset{!}{=} 0\\
\Leftrightarrow	~~	& 7\theta^6 - 12 \theta^5 + 5 \theta^4 &= 0\\
\Leftrightarrow	~~	& 7\theta^6 - 12 \theta^5 + 5 \theta^4 &= 0\\
\Leftrightarrow	~~	& \theta^4(7\theta^2 - 12 \theta + 5 \theta) &= 0\\
\Leftrightarrow	~~	& \theta^2 - \frac{12}{7} \theta + \frac{5}{7} \theta &= 0\\
\Leftrightarrow	~~	& \theta_{1,2} = \frac{6}{7} \pm \sqrt{\frac{36}{49} - \frac{35}{49}} \\
\Leftrightarrow	~~	& \theta_{1,2} = \frac{6}{7} \pm \frac{1}{7} \\
\Leftarrow	~~	& \theta_{1} = \frac{5}{7} ~~ \text{and} ~~ \theta_{2} =1 \\
\end{aligned}
\end{align}

From the fourth line of equation \ref{eq:condmaximum}, $\theta_0 = 0$ could be solution. Further the possible solutions $ \theta_{1} = \frac{5}{7}$ and $\theta_{2} =1$. Taking a look a t the second derivation:

\begin{align}
\begin{aligned}
\frac{\partial^2 P(\mathcal{D}|\theta)}{\partial \theta \partial \theta} = 42\theta^5 - 60 \theta^4 + 20 \theta^3
\end{aligned}
\end{align}

For $\theta_0 = 0$ the second derivation is also zero. At first the fifth derivation is not zero and thus, $\theta_0 = 0$ is a turning point. For $\theta_1 = \frac{5}{7}$ the second derivtaion is negative, so  $\theta_1 = \frac{5}{7}$ is a maximum. For $\theta_2 = 1$ the second derivtaion is positive,  $\theta_2$ is a minimum. This leads to $\hat\theta = \frac{5}{7}$.\\
\\
The coin tosses are independent, so $P(x_8 = \text{head}, x_9 = \text{head} | \hat\theta) = P(x_8 = \text{head}|\hat\theta) \cdot P(x_9 =\text{head}|\hat\theta)$. Thus:

\begin{equation}
P(x_8 = \text{head}, x_9 = \text{head} | \hat\theta) = \hat\theta^2 = \frac{25}{49}
\end{equation}

\subsection*{Task c)}

\textit{We now adopt a Bayesian view on this problem, where we assume a prior distribution for the parameter $\theta$ defined as:}

\begin{equation}
p(\theta) = \left\{
\begin{array}{ll} 1 	& \text{if} ~ 0 \leq \theta \leq 1\\
  			 0	&  \text{else}
\end{array}\right.
\end{equation}

\textit{Compute the posterior distribution $p(\theta|\mathcal{D})$, and evaluate the probability that the next two tosses are head, that is,}

\begin{equation}\label{eq:doubleheadtossint}
\int P(x_8 = \text{head}, x_9 = \text{head} | \theta) p(\theta|\mathcal{D}) d\theta
\end{equation}

First  $p(\theta|\mathcal{D})$ is calculated:

\begin{align}
\begin{aligned}
p(\theta|\mathcal{D}) & = \frac{p(\mathcal{D}|\theta)p(\theta)}{p(\mathcal{D})}\\
				& = \frac{p(\mathcal{D}|\theta)p(\theta)}{\int p(\mathcal{D}|\theta)p(\theta)d\theta}\\
				& = \frac{\theta^5 \cdot (1 - \theta)^2 \cdot 1}{\int \theta^5 \cdot (1 - \theta)^2 \cdot 1 d\theta}\\
				& = \frac{\theta^5 - 2 \theta ^6 + \theta^7}{\int \theta^5 - 2 \theta ^6 + \theta^7 d\theta}\\
				& = \frac{\theta^5 - 2 \theta ^6 + \theta^7}{\left[\frac{1}{6}\theta^6 - \frac{2}{7} \theta ^7 + \frac{1}{8}\theta^8\right]_0^1}\\
				& = \frac{\theta^5 - 2 \theta ^6 + \theta^7}{\frac{1}{168}}\\
				& = 168(\theta^5 - 2 \theta ^6 + \theta^7)\\
\end{aligned}
\end{align}
\newpage
Inserted into equation \ref{eq:doubleheadtossint}:

\begin{align}
\begin{aligned}
	& \int P(x_8 = \text{head}, x_9 = \text{head} | \theta) p(\theta|\mathcal{D}) d\theta\\
=	& \int \theta^2 \cdot 168(\theta^5 - 2 \theta ^6 + \theta^7) d\theta\\
=	& \int 168(\theta^7 - 2 \theta ^8 + \theta^9) d\theta\\
=	& \left[ 168 \cdot(\frac{1}{8}\theta^8 - \frac{2}{9} \theta ^9 + \frac{1}{10}\theta^{10}) \right]_0^1\\
=	& 168 \cdot (\frac{1}{8} - \frac{2}{9} + \frac{1}{10})\\
=	& \frac{7}{15}
\end{aligned}
\end{align}

%----------------------------------------------------------------------------------------
%	SECTION 2
%----------------------------------------------------------------------------------------

\section{Convergence of Bayes Parameter Estimation}

%----------------------------------------------------------------------------------------
%	SECTION 3
%----------------------------------------------------------------------------------------
\subsection*{(a)}
Show that $ \sigma^2_n \leq \min \left (\frac{n}{\sigma^2},\sigma^2_0 \right )$.
\subsubsection*{Assumption (I)}
From Chapter 3.4.1 in Duda et al we know, that $n > 0$.\\
Furthermore we have obviously $\sigma^2 > 0, ~\sigma^2_0 > 0$ and $\sigma^2_n > 0$, since these are squares of the standard deviation. Because the $\sigma$'s are quotients in the given formula, we know that they must be bigger than $0$.\\
\subsubsection*{Proof}
On the one hand we have
\begin{align*}
				   &\frac{1}{\sigma^2_n} = \frac{n}{\sigma^2} + \frac{1}{\sigma^2_0}&\\
\Leftrightarrow ~~~&\frac{1}{\sigma^2_n} - \frac{1}{\sigma^2_0} = \frac{n}{\sigma^2}&\\
\Leftrightarrow ~~~&\frac{\sigma^2_0 - \sigma^2_n}{\sigma^2_n\sigma^2_0} = \frac{n}{\sigma^2} \stackrel{\text{(I)}}{\geq} 0&\\
\Leftrightarrow ~~~&\frac{\sigma^2_0 - \sigma^2_n}{\sigma^2_n\sigma^2_0} \geq 0&\\
\Leftrightarrow ~~~&\sigma^2_0 \geq \sigma^2_n& \text{, because } \sigma^2_n\sigma^2_0 > 0 \text{ using (I)}\\
\end{align*}
And on the other hand
\begin{align*}
				   &\frac{1}{\sigma^2_n} = \frac{n}{\sigma^2} + \frac{1}{\sigma^2_0}&\\
\Leftrightarrow ~~~&\frac{1}{\sigma^2_n} - \frac{n}{\sigma^2} = \frac{1}{\sigma^2_0}&\\
\Leftrightarrow ~~~&\frac{\sigma^2 - n\sigma^2_n}{\sigma^2_n\sigma^2} = \frac{1}{\sigma^2_0} \stackrel{\text{(I)}}{\geq} 0&\\
\Leftrightarrow ~~~&\frac{\sigma^2 - n\sigma^2_n}{\sigma^2_n\sigma^2} \geq 0&\\
\Leftrightarrow ~~~&\sigma^2 \geq n\sigma^2_n& \text{, because } \sigma^2_n\sigma^2 > 0 \text{ using (I)}\\
\Leftrightarrow ~~~&\frac{\sigma^2}{n} \geq \sigma^2_n& \text{, because } n > 0 \text{ using (I)}~~~~~\\
\end{align*}
So we have shown, that $\sigma^2_n \leq \frac{\sigma^2}{n}$ and $\sigma^2_n \leq \sigma^2_0$ and it follows
$$
\sigma^2_n \leq \min \left (\frac{n}{\sigma^2},\sigma^2_0 \right ).
$$
\hspace*{\fill}$\Box$\\
\subsection*{(b)}
Show that $\min(\hat{\mu}_n,\mu_0) \leq \mu_n \leq \max(\hat{\mu}_n,\mu_0)$\\ \\
In the following proof a reference to (I) refers to assumption (I) from exercise 2 (a).\\
\subsubsection*{Proof}
\underline{Case 1: $\mu_n = 0$}\\
In this case we have 
$$ \frac{\mu_n}{\sigma^2_n} = 0 = \frac{n}{\sigma^2}\hat{\mu}_n + \frac{1}{\sigma^2_0}\mu_0 $$
With (I) we have  $\frac{1}{\sigma^2_0} > 0$ and $\frac{n}{\sigma^2} > 0$ and since the sum in the last part of the equation equals $0$ it follows, that 
$$ \min(\hat{\mu}_n,\mu_0) \leq 0  \leq \max(\hat{\mu}_n,\mu_0) \stackrel{\mu_n = 0}{\Leftrightarrow} \min(\hat{\mu}_n,\mu_0) \leq \mu_n \leq  \max(\hat{\mu}_n,\mu_0)$$
\underline{Case 2: $\mu_n \neq 0$}\\
We have
\begin{align*}
	             &\frac{\mu_n}{\sigma^2_n} = \frac{n}{\sigma^2}\hat{\mu}_n + \frac{1}{\sigma^2_0}\mu_0&\\
\Leftrightarrow~~&\frac{1}{\sigma^2_n} = \frac{n}{\sigma^2} \cdot \frac{\hat{\mu}_n}{\mu_n} + \frac{1}{\sigma^2_0} \cdot \frac{\mu_0}{\mu_n}&\\
\Leftrightarrow~~&\frac{n}{\sigma^2} + \frac{1}{\sigma^2_0} = \frac{n}{\sigma^2} \cdot \frac{\hat{\mu}_n}{\mu_n} + \frac{1}{\sigma^2_0} \cdot \frac{\mu_0}{\mu_n}& \text{, because } \frac{1}{\sigma^2_n} = \frac{n}{\sigma^2} + \frac{1}{\sigma^2_0}\\
\Leftrightarrow~~&\frac{n}{\sigma^2} - \frac{n}{\sigma^2} \cdot \frac{\hat{\mu}_n}{\mu_n}  =   \frac{1}{\sigma^2_0} \cdot \frac{\mu_0}{\mu_n} - \frac{1}{\sigma^2_0}& \\
\Leftrightarrow~~&\frac{n}{\sigma^2} \cdot \frac{\mu_n - \hat{\mu}_n}{\mu_n}  =   \frac{1}{\sigma^2_0} \cdot \frac{\mu_0 - \mu_n}{\mu_n}& \\
\Leftrightarrow~~&\frac{n\sigma^2_0}{\sigma^2} \cdot \frac{\mu_n - \hat{\mu}_n}{\mu_n}  =   \frac{\mu_0 - \mu_n}{\mu_n}& 
\end{align*}  
\underline{Case 2.1: $\mu_n - \hat{\mu}_n = 0$}\\
Then we have
$$\mu_n = \hat{\mu}_n$$ 
and 
$$0 = \frac{\mu_0 - \mu_n}{\mu_n} \Leftrightarrow  \mu_0 = \mu_n$$
and it follows
$$(\mu_0 = \mu_n = \hat{\mu}_n) \Rightarrow (\mu_0 \leq \mu_n \leq \hat{\mu}_n) \Rightarrow (\min(\hat{\mu}_n,\mu_0) \leq \mu_n \leq  \max(\hat{\mu}_n,\mu_0))$$ 
\underline{Case 2.2: $\mu_n - \hat{\mu}_n \neq 0$}\\
Then we have
\begin{align*}
&\frac{n\sigma^2_0}{\sigma^2} \cdot \frac{\mu_n - \hat{\mu}_n}{\mu_n}  =   \frac{\mu_0 - \mu_n}{\mu_n}& \\
\Leftrightarrow~~&\frac{n\sigma^2_0}{\sigma^2} =   \frac{\mu_0 - \mu_n}{\mu_n - \hat{\mu}_n}& 
\end{align*}
With (I) we get
$0 \leq \frac{n\sigma^2_0}{\sigma^2}$ 
which gives us
$$0 \leq \frac{\mu_0 - \mu_n}{\mu_n - \hat{\mu}_n}$$
To fullfill this inequation must either  be 
$$(0 \leq \mu_0 - \mu_n) \wedge (0 \leq \mu_n - \hat{\mu}_n)$$ 
or 
$$(0 \geq \mu_0 - \mu_n) \wedge (0 \geq \mu_n - \hat{\mu}_n)$$
In the first case we have
\begin{align*}
                 &(0 \leq \mu_0 - \mu_n) \wedge (0 \leq \mu_n - \hat{\mu}_n)&\\
\Leftrightarrow~~&(\mu_n \leq \mu_0 ) \wedge (\hat{\mu}_n \leq \mu_n )&\\
\Leftrightarrow~~&(\hat{\mu}_n \leq \mu_n ) \wedge (\mu_n \leq \mu_0 )&\\
\Leftrightarrow~~&\hat{\mu}_n \leq \mu_n \leq \mu_0 &\\
\Leftrightarrow~~& \min(\hat{\mu}_n,\mu_0) \leq \mu_n \leq  \max(\hat{\mu}_n,\mu_0)& 
\end{align*}
The second case is analogous to the first.\\
\hspace*{\fill}$\Box$
\section{Programming}

%----------------------------------------------------------------------------------------
%	BIBLIOGRAPHY
%----------------------------------------------------------------------------------------

%\bibliographystyle{apalike}

%\bibliography{sample}

%----------------------------------------------------------------------------------------


\end{document}